\documentclass{../../../../style/mkimain}
\series{3}
\month{duben}
\year{2023}

\begin{document}
%<*header>
\section*{III.B Weyl vs. Majorana: boj o neutrino}
%</header>
%<*task>
\noindent
Během dvacátých a třicátých let 20. století vznikla spousta kvantově mechanických rovnic na popis různých typů fermionů. Mezi ně patří i tzv.
Weylova a Majoranova rovnice, které dříve byly kandidáty na popis částice jménem \emph{neutrino}. Pojďme se podívat, jak vypadají!
\\
\\
\textbf{Pozn.:} ve vzorcích níže je použita Einsteinova sumační konvence, Feynmanova \uv{slash} notace $\partial\!\!\!/=\gamma^\mu \partial_\mu$
a standardní volba jednotek $\hbar=c=1$.
\begin{enumerate}
    \item Odvoďte Weylovu rovnici (rovnice), popisující nehmotné (Weylovy) fermiony, ve slavném tvaru
    $$
    \sigma^\mu \partial_\mu \psi_R = 0
    $$
    $$
    \bar{\sigma}^\mu \partial_\mu \psi_L = 0 \text{.}
    $$
    $\psi_L$ značí levoruký a $\psi_R$ pravoruký Weylův spinor a vektory $\sigma^\mu$ a $\bar{\sigma}^\mu$ jsou definované jako
    $$\sigma^\mu = \left(\sigma^0, \sigma^1, \sigma^2, \sigma^3\right) = \left(I_2, \sigma_x, \sigma_y, \sigma_z\right)$$
    $$\bar{\sigma}^\mu = \left(\sigma^0, -\sigma^1, -\sigma^2, -\sigma^3\right)\text{,}$$
    Kde první komponent $\sigma^0=I_2$ je jednotková matice typu $2\times2$ a
    zbylé složky obsahují Pauliho spinové matice ($\sigma^i\text{,}\;i\in \{1,2,3\}$).
    
    \item Matematicky dokažte, že rovnice
    $$i\partial\!\!\!/\psi^c-m\psi=0$$
    je ekvivalentní s Majoranovu rovnicí, která bývá psána jako
    $$i\partial\!\!\!/\psi-m\psi^c=0\text{,}$$
    kde $m$ označuje hmotnost popisovaného fermionu a $\psi$ jeho vlnovou funkci. Horní index $c$ značí nábojové sdružení.
\end{enumerate}
%</task>
\end{document}
