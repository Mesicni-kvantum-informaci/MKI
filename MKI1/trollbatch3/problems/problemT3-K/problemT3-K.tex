\documentclass{../../../../style/mkimain}
\usepackage{verse}
\series{3}
\month{duben}
\year{2023}

\begin{document}
%<*header>
\section*{III.K Diracovo moře}
%</header>
\begin{verse}
Prostorem se vlnový balík řítí,\\
občas zrychluje a s tím trochu svítí.\\
Okolní fotony ho poznají,\\
Však on svůj náboj vůbec netají.\\
Jejich vlnová délka hlásá: „je to on!“\\
Všem gaugovým částicím známý elektron…
\end{verse}
\begin{verse}
Po rovnici fermionů se Dirac pídí,\\
snaží se zjistit, čím se elektrony řídí.\\
Gama matice použije,\\
spoustu slávy si pak užije.\\
Je tu však jeden zádrhel,\\
elektron je fakt vyvrhel…
\end{verse}
\begin{verse}
Diracova rovnice má dvě řešení,\\
avšak na našem světě se nic nemění.\\
Jedno odpovídá elektronům,\\
druhé naopak jiným démonům,\\
„elektronům“ se zápornou energií.\\
Celou dobu si v tomto vesmíru žijí!
\end{verse}
\begin{verse}
Společně tak tvoří pole\\
s energií hodně dole.\\
Sem tam mají nějakou díru,\\
až překopávám svoji víru.\\
Tato díra, to jen on,\\
již všem známý pozitron!
\end{verse}
\begin{verse}
Jednou však takhle Diraca napadne,\\
co když elektron do díry zapadne?\\
Dvě částice se přitom uvolní,\\
svým charakterem částice polní.\\
Ano, jsou to opravdu ony.\\
Ty známé Planckovy fotony.
\end{verse}
\begin{verse}
To byl příběh o tom, jak\\
Paul Dirac všem vytřel zrak.\\
Nové částice tak předpověděl,\\
záhady vesmíru tím zodpověděl.\\
Dnes ho známe jak své boty,\\
první model antihmoty.\\
\vspace{0.5cm}
\textbf{Úloha:}
\\
%<*task>
Jaký jev je popisován v páté sloce?
%</task>
\end{verse}
\end{document}
