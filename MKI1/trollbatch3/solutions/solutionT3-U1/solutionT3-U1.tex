\documentclass{../../../../style/mkimain}

\series{3}
\month{duben}
\year{2023}

\begin{document}
\ExecuteMetaData[../../problems/problemT3-U1/problemT3-U1.tex]{header}
\noindent\ExecuteMetaData[../../problems/problemT3-U1/problemT3-U1.tex]{task}
\proborigin{Michal má rád žhavou (ro)ma(n)tiku.}
\klein
V roce 1843 u irského mostu Broom Bridge,
zatímco se na něj jeho manželka dívala romantickým pohledem žádajícím o vášnivý polibek nad poklidně šumící řekou,
sir William Rowan Hamilton dostal nápad, jak úspěšně rozšířit matematický obor komplexních čísel.
Rozhodl se přidat hned dvě další imaginární jednotky namísto jedné. Do mostu, kolem něhož se svou milou procházel,
vyryl soustavu rovnic, která imaginární jednotky jeho nového číselného oboru jednoznačně definuje.
$$i^2=j^2=k^2=ijk=-1$$
Nově přidané jednotky jsou tedy $j$ a $k$. Čísla, která jsou tvořena těmito jednotkami, dostala název \emph{kvaterniony} a to kvůli tomu,
že sestávají celkem ze čtyř částí, jedné reálné a tří imaginárních. Od reálných a komplexních čísel se liší jednou zajímavou vlastností, kterou si nyní ukážeme.
\\
\\
Z rovnice výše si vytáhneme rovnici
$$ijk=-1$$
a vynásobíme obě strany imaginární jednotkou $k$ (zprava).
$$ijkk=-k$$
$$ijk^2=-k$$
Evidentně dle původní definice $k^2=-1$.
$$-ij=-k$$
Pokračujeme dál\dots Vynásobíme poslední rovnici $i$ zleva.
$$-i^2j=-ik$$
$$j=-ik$$
Nově vyjádřené $j$ dosadíme za jedno $j$ v rovnici
$$j^2=ijk\text{.}$$
$$j(-ik)=ijk$$
$$-jik=ijk$$
A vynásobíme $k$ zprava.
$$-jik^2=ijk^2$$
$$ji=-ij$$
Dostáváme vskutku zajímavý výsledek. Na pořadí násobení záleží. Proto po celou dobu rozboru je zdůrazňováno, zda se jedná o násobí zleva, nebo zprava.
U kvaternionů na tom vskutku sejde. Neplatí u nich totiž komutativnost násobení. Každopádně dostáváme vhodné kandidáty na čísla $a$ a $b$.
Ještě musíme ověřit jejich absolutní hodnotu, respektive absolutní hodnotu jejich druhé mocniny. Druhá mocnina obou imaginárních jednotek je rovna $-1$ a absolutní hodnota $-1$ je rovna $1$.
Našli jsme tedy správná čísla.
$$a=i$$
$$b=j$$
Samozřejmě existují ještě další kvaterniony, které zadané podmínky splňují, ale nám postačí $i$ a $j$. Krom toho, i kvaterniony byly o chvíli později rozšířeny na komplexnější obor čísel, \emph{oktoniony},
které mají celkem osm částí, posléze i na \emph{sedeniony} se šestnácti částmi a úplně závěrem i na \emph{trigintaduoniony}, které mají rekordních 32 složek. Souhrně jsou kvaterniony a všechny tyto složitější skupiny čísel
nazývány \emph{hyperkomplexní čísla}. Ovšem to nejsou v matematice jediná čísla splňující naše podmínky. Kupříkladu vektorový součin dvou vektorů je antikomutativní, jenže zase nekoresponduje s notací standardního násobení,
při kterém se dá vynechat znak násobení. Objekty, u kterých to však funguje, jsou zase o něco složitější a nazývají se \emph{matice}.
Jedná se o soubor čísel uspořádaných do řádků a sloupců, který sice v rovnici může reprezentovat soustavu rovnic, ovšem je klasifikován jako samostatná proměnná. Má speciální pravidla pro násobení, která také nezaručují komutativitu.
Mezi maticemi bychom mohli také hledat řešení. Dokonce i sám Paul Dirac, britský kvantový fyzik, využil vlstnosti jejich násobení a postavil na nich svou slavnou Diracovu rovnici pro relativistický popis fermionů.
\end{document}
