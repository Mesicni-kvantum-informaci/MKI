\documentclass{../../../../style/mkimain}

\series{3}
\month{duben}
\year{2023}

\begin{document}
\ExecuteMetaData[../../problems/problemT3-K/problemT3-K.tex]{header}
\noindent\ExecuteMetaData[../../problems/problemT3-K/problemT3-K.tex]{task}
\proborigin{Michal přemýšlel nad Diracovým mořem a skoro se v něm utopil.}
\klein
Pátá sloka vypráví o jevu, kdy elektron \uv{zapadne} do pomyslné díry v Diracově moři.
Tato díra je v Diracově teorii intepretována jako pozitron. Když do ní elektron zapadne,
z našeho pohledu zmizí a zároveň zaplní díru, která pro nás tudíž také nebude existovat. Výsledkem je,
že elektron a pozitron (díra) se komplentě vymažou z našeho světa
a uvolní přitom veškerou svou energii v podobě záření (fotonů). Tento jev se nazývá \emph{anihilace} (elektronu s pozitronem).

\end{document}
