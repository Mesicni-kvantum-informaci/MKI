\documentclass{../../../../style/mkimain}

\series{3}
\month{duben}
\year{2023}

\begin{document}
\ExecuteMetaData[../../problems/problemT3-B/problemT3-B.tex]{header}
\noindent\ExecuteMetaData[../../problems/problemT3-B/problemT3-B.tex]{task}
\proborigin{Michal chtěl flexit se svými znalostmi QFT.}
\klein

\begin{enumerate}
    \item Pro potřeby tohoto řešení můžeme použít formu klasického bispinoru.
    \begin{equation} \label{eq:1}
    \psi=\begin{pmatrix}
        \psi_L\\
        \psi_R
    \end{pmatrix}
    \end{equation}
    Kde složky vlnové funkce $\psi_L$ a $\psi_R$ nám určují chiralitu. Tato forma odpovídá popisu a interpretaci německého fyzika Hermanna Weyla, který se chiralitou částic hojně zabýval a vytvořil pro svoje teoretické úvahy vlastní reprezentaci gama matic, tzv. \emph{Weylovu chirální}. V ní matice $\gamma^1$, $\gamma^2$ a $\gamma^3$ vypadají úplně stejně jako Diracovy, pouze matice $\gamma^0$ má tvar
     $$\gamma^0=\begin{pmatrix}
        0 & I_2\\
        I_2 & 0
    \end{pmatrix} \text{.}$$
A pro všechna $i\in\{1;2;3\}$ mají gama matice podobu
     $$\gamma^i=\begin{pmatrix}
        0 & \sigma^i\\
        -\sigma^i & 0
    \end{pmatrix} \text{.}$$
Všimněme si, že $\gamma^0$ matice obsahuje $I_2=\sigma^0$ a že jsou tyto prvky na stejných pozicích, jako spinové matice v dalších gama maticích. Když jsme na začátku definovali Pauliho spinové matice jako čtyřvektor $\sigma^\mu$ a jeho sdruženou podobu $\bar{\sigma}^\mu$, můžeme docela chytře zapsat Weylovy gama matice rovněž v úsporném čtyrvektrovém tvaru.
\begin{equation} \label{eq:2}
 \gamma^\mu=\begin{pmatrix}
        0 & \sigma^\mu\\
        \bar{\sigma}^\mu & 0
    \end{pmatrix}
\end{equation}
Přičemž $\mu\in\{1;2;3;4\}$.
Tuto reprezentaci gama matic použijeme při odvození Weylových rovnic pro nehmotné fermiony. Obecně když jde stále o fermiony, musí vše vycházet z Diracovy rovnice, která je přeci jenom popisuje v plné kráse.
 $$\left(i\partial\!\!\!/-m\right)\psi=0$$
Tím, že jsou hypotetické Weylovy fermiony nehmotné ($m=0$), se Diracova rovnice redukuje na tvar
 \begin{equation}\label{eq:3}
 \partial\!\!\!/\psi=0\text{.}    
 \end{equation}
 Na základě Feynmanovy \uv{slash} notace víme, že rozepsaný tvar rovnice vypadá takto:
$$\gamma^\mu\partial_\mu\psi=0\text{.}$$
Následně použijeme vztahy \ref{eq:1} a \ref{eq:2} z Weylovy chirální reprezentace.
$$ \begin{pmatrix}
        0 & \sigma^\mu\\
        \bar{\sigma}^\mu & 0
    \end{pmatrix} \partial_\mu \begin{pmatrix}
        \psi_L\\
        \psi_R
    \end{pmatrix}=0 $$
A roznásobíme\dots
$$ \begin{pmatrix}
        0 & \sigma^\mu\\
        \bar{\sigma}^\mu & 0
    \end{pmatrix} \begin{pmatrix}
        \partial_\mu\psi_L\\
        \partial_\mu\psi_R
    \end{pmatrix}=0 $$
$$ \begin{pmatrix}
        \sigma^\mu \partial_\mu\psi_R\\
        \bar{\sigma}^\mu \partial_\mu\psi_L
    \end{pmatrix}=0 $$
    Což jsou už v podstatě Weylovy rovnice pro nehmotné fermiony zapsané jen pomocí vektoru.
    $$
    \sigma^\mu \partial_\mu \psi_R = 0
    $$
    $$
    \bar{\sigma}^\mu \partial_\mu \psi_L = 0
    $$
    Je zajímavé ohlédnout se do historie a zajímat se o to, co se o neutrinech domnívalo dříve. Dlouhou dobu bylo lidstvo přesvědčeno, že neutrina skutečně patří do skupiny částic s nulovou klidovou hmotností, což by rovněž implikovalo fakt, že je jejich rychlost přesně světelná (jak lze dokázat z rovnice \ref{eq:3}). Tuto informaci zahrnoval standardní model částic dost dlouhou dobu. Dnes však víme, že byla chybná. Ale s podobnou jistotou prohlašuje Richard Feynman ve svých přednáškách\footnote{FEYNMAN, Richard P., LEIGHTON, Robert B., SANDS, Matthew. \textit{The Feynman Lectures on Physics 3}. California Institute of Technology, USA: Addison-Wesley Longman, 1965.}, že neutrina skutečně klidovou hmotnost nemají. V současné době ale naše experimenty spíše potvrzují, že jim určitou klidovou hmotnost připsat můžeme, sice dost malou, ale můžeme. Neutrina jsou tedy z dnešního pohledu lehounké a zároveň dost rychlé částice. Jsou dokonce tak rychlé, že v roce 2011 jim byla v CERNu naměřená rychlost vyšší než rychlost světla. Ve skutečnosti se však jednalo jen o technickou chybu, ale poukazuje to na to, jak málo toho o neutrinech víme a jak málo toho jsme schopni o těchto prazvláštních částicích zjistit.

    Důkaz, že neutrina jsou opravdu hmotné částice, přineslo až pozorování jevu zvaného \emph{oscilace neutrin}. Ten připouští, že jednotlivé typy neutrin mají i nějakou malou pravděpodobnost, že ponesou hmotnost jiného typu neutrina. To jim umožňuje přeměňovat se mezi sebou, vydávat se za jiná neutrina. Ve standardním modelu částic existují tři, elektronové, mionové a tauonové. Každé z nich má řádově odlišnou hmotnost, což je předpoklad k tomu, aby oscilace probíhala. Kdyby byla neutrina nehmotná, nic takového bychom pozorovat ani nemohli. Tento jev tedy vyvrátil domněnku, že neutrina patří mezi Weylovy fermiony.
    
    \item
    Už jsme si udělali jasno v tom, jakou hmotnost a rychlost neutrina mají. Prozradím vám, že toto nebyla jediná neutrinová záhada, která lidstvu vrtala hlavou. Dodnes totiž nevíme, jestli náhodou nejsou neutrina tzv. \emph{Majoranovy fermiony}. Fermiony, které jsou samy sobě antičásticí.
    
    Existuje matematická operace, která nám umí převést vlnovou funkci částice na vlnovou funkci její příslušné antičástice, která až na náboj má všechny stejné výchozí podmínky. Říká se jí nábojové sdružení a má tvar
    $$\psi ^c=C\overline{\psi}^\mathsf{T}\text{.}$$
    $C$ je matice nábojového sdružení. Ta má v Diracově reprezentaci (po zbytek úlohy budeme používat už jen jeho) tvar
    $$C=\begin{pmatrix}
        0 & -i\sigma_y\\
        -i\sigma_y & 0
    \end{pmatrix} \text{.}$$
    Řekněme tedy, že máme volnou (interakce s vnějšími poli a ostatními částicemi nás zatím nezajímá) částici s vlnovou funkcí $\psi$. Sama se bude řídit Diracovou rovnicí. 
 $$\left(i\partial\!\!\!/-m\right)\psi=0$$
Nyní vytvořme k dané částici její antičástici, stále ve stejných podmínkách. Její vlnová funkce bude $\psi^c$. Jelikož antičástice fermionů jsou opět fermiony, musí i naše antičástice splňovat Diracovu rovnici.
$$\left(i\partial\!\!\!/-m\right)\psi^c=0$$
Nyní si s rovnicemi můžeme libovolně hrát. Zkusme je například sečíst.
$$\left(i\partial\!\!\!/-m\right)\psi+\left(i\partial\!\!\!/-m\right)\psi^c=0$$
$$i\partial\!\!\!/\psi-m\psi+i\partial\!\!\!/\psi^c-m\psi^c=0$$
$$\left(i\partial\!\!\!/\psi^c-m\psi\right)+\left(i\partial\!\!\!/\psi-m\psi^c\right)=0$$
Pokud má ze zadání platit, že
    $$i\partial\!\!\!/\psi^c-m\psi=0\text{,}$$
musí i druhý výraz být roven nule.
$$i\partial\!\!\!/\psi-m\psi^c=0$$
Poměrně prostá matematika, že? Je to docela hezký argument, ale na začátku jsme vyslovili předpoklad, že i antičástice splňuje Diracovu rovnici. Co kdybychom se obešli bez toho a dopracovali se k výsledku čistou matematikou?
Začněme s rovnicí
     \begin{equation}\label{eq:4}
    i\partial\!\!\!/\psi^c-m\psi=0\text{.}
    \end{equation}
    Rozepišme nábojové sdružení.
$$\psi^c=C\overline{\psi}^\mathsf{T}=C\left(\psi^\dagger\gamma^0\right)^\mathsf{T}=C\left(\gamma^0\right)^\mathsf{T}\left(\psi^\dagger\right)^\mathsf{T}=C\gamma^0\psi^*=-\gamma^0C\psi^*$$    
To, že matice nábojového sdružení antikomutuje s gama nula maticí, se může dokázat prostým násobením těchto matic v jednom pořadí a pak v druhém. Náš získaný tvar nábojového sdružení dosadíme do rovnice \ref{eq:4} a přenásobíme $-1$.
$$i\partial\!\!\!/\left(\gamma^0C\psi^*\right)+m\psi=0$$
Vynásobíme rovnici zleva $\gamma^0$. A rozepíšeme $\partial\!\!\!/$ na $\gamma ^\mu \partial_\mu$.

$$i\gamma^0\gamma^\mu\gamma^0C\partial_\mu \psi^*+m\gamma^0\psi=0$$
Opět pracným násobením matic si můžete dokázat identitu
$$\gamma^0\gamma^\mu\gamma^0=\left(\gamma^\mu\right)^\dagger \text{,}$$
kterou využijeme na zkrácení zápisu.
$$i\left(\gamma^\mu\right)^\dagger C\partial_\mu \psi^*+m\gamma^0\psi=0$$
Dále rovnici komplexně sdružíme.
$$-i\left(\gamma^\mu\right)^\mathsf{T} C\partial_\mu \psi+m\gamma^0\psi^*=0$$
Pokud si vzpomínáte, tak $\gamma^0\psi^*=\overline{\psi}^\mathsf{T}$. Vynásobme rovnici zleva $C$.
$$-iC\left(\gamma^\mu\right)^\mathsf{T} C \partial_\mu \psi+mC\overline{\psi}^\mathsf{T}=0$$
Další vztah, který si lze dokázat pronásobováním matic, je
$$C\left(\gamma^\mu\right)^\mathsf{T} C=\gamma^\mu\text{.}$$
A pokud užijeme definici nábojového sdružení, zjistíme, že se rovnice ještě více zredukuje.
$$-i\gamma^\mu\partial_\mu \psi+m\psi^c=0$$
Přenásobením $-1$ a užitím Feynmanovy \uv{slash} notace se dostaneme finálně k Majoranově rovnici.
$$i\partial\!\!\!/\psi-m\psi^c=0$$
Zda neutrina tuto rovnici splňují, dnes stále nikdo neví. Velký problém totiž tkví v experimentálním důkazu. Zatím nevíme, jaký experiment provést, abychom mohli Majoranovu hypotézu v případě neutrin potvrdit nebo vyvrátit. Jeden z nejdiskutovanějších, teoreticky možných experimentů je pozorování \emph{dvojitého beta rozpadu bez neutrin}.
Vezměme si kupříkladu beta mínus rozpad.
   $$\ce{n -> p + e- + \overline{\nu}_e}$$
Neutron se rozpadá na proton, elektron a antineutrino. Co když proběhnou dva tyto rozpady téměř ve stejnou chvíli?$$\ce{2n -> 2p + 2e- + 2\overline{\nu}_e -> 2p + 2e-}$$
Zkrátka nám vzniknou dvě antineutrina. Pokud jsou neutrina opravdu Majoranovy fermiony, jsou totožné s antineutriny a anihilují navzájem, ačkoliv se jedná o částice stejného druhu.
Při dvojitém beta rozpadu je možné, aby dvě neutrina (antineutrina), která vznikají jako produkt, navzájem anihilovala už v jakémsi vnitřním mechanismu celého rozpadu. Jako výsledek bychom pozorovali pouze dva protony a dva elektrony.

Problém s tímto experimentálním provedením je, že neutrina jsou částice, které prochází téměř vším a je velmi nepravděpodobné jejich zachycení a detekce. Pokud tedy při experimentu s dvojitým beta rozpadem žádná neutrina nezpozorujete, bude to pravděpodobně tím, že vám proklouzla mezi prsty a ne tím, že anihilovala. Jako řešení tohoto problému se jeví detekovat produkty z jejich samotné anihilace, ale i to je poněkud komplikované a ne tolik vypovídající. Neutrina tak zůstávají pro lidstvo stále ještě obrovskou záhadou\dots
\end{enumerate}
\end{document}
