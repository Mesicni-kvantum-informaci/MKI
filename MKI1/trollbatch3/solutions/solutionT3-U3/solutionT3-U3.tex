\documentclass{../../../../style/mkimain}

\series{3}
\month{duben}
\year{2023}

\begin{document}
\ExecuteMetaData[../../problems/problemT3-U3/problemT3-U3.tex]{header}
\noindent\ExecuteMetaData[../../problems/problemT3-U3/problemT3-U3.tex]{task}
\proborigin{Michalovi už z toho věčného psaní vzoráků hrabe.}
\klein
Probereme si postupně zvláštnost každého vědce. Nikola Tesla se dle všech zvěstí zamiloval do holuba,
ergo ho můžeme klasifikovat jako zoofila. O Paulu Diracovi se traduje, že mluvil tempem jedno slovo za hodinu.
Byl nesmírně uzavřený a nespolečenský, jelikož měl Aspergerův syndrom. Albert Einstein proslul svým citátem:
\begin{center}
\emph{\uv{Nic nebude lidskému zdraví prospěšnější a nic nezvýší šance na zachování života na Zemi více než přechod na vegetariánskou stravu.}}
\end{center}
Jednoznačný vegetarián. Erwin Schrödinger, když zrovna nehledal řešení jeho rovnice pro různá kvantová čísla v atomu vodíku,
údajně zavíral děti do svého sklepa a prováděl na nich ty své neintelektuální potřeby. Bernhard Riemann za život nepromluvil snad ani slovo na veřejnosti.
Kdežto William Rowan Hamilton toho asi napovídal hodně, když byl celý život namočený v irské. Isaac Newton to asi nikdy s žádnou ženou nerozjel.
Jediné, co na něj v životě hupslo, bylo možná tak to slavné jablko.
Alan Turing si to taky mockrát nezkusil, zato mladými chlapci jistě nepohrdl.
A Emmy Noether na tom nejspiše byla podobně, poněvadž byla sama žena.
\\
\\
Závěrem by se hodilo podotknout, že mnoha poruchami trpělo i vícero lidí a mnoho lidí mělo vícero poruch. Možností bylo tedy spávných více.
Jak už víte, fyzici jsou opravdu cáklí!
\end{document}
