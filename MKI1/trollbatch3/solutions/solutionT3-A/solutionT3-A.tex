\documentclass{../../../../style/mkimain}

\series{3}
\month{duben}
\year{2023}

\begin{document}
\ExecuteMetaData[../../problems/problemT3-A/problemT3-A.tex]{header}
\noindent\ExecuteMetaData[../../problems/problemT3-A/problemT3-A.tex]{task}
\proborigin{Michal a Vojta si hráli s ChatGPT.}
\klein

\noindent Autorem \emph{Problému tří těles} je čínský spisovatel Liou Cch’-sin. Děj knihy vypadá přibližně takto:

\begin{center}
\emph{\uv{V Říši středu zuří Velká kulturní revoluce a Číňané, kteří nechtějí zůstat pozadu za Sověty a Američany, se v rámci tajného vojenského projektu
pokoušejí navázat kontakt s mimozemskými civilizacemi. Třicet let poté začnou na Zemi umírat významní vědci a vznikají sekty, které
vybízejí k návratu k přírodě. Objeví se nová počítačová hra pro virtuální realitu, zvláštní, čarovná a znepokojivá, která jako by ani
nepocházela z tohoto světa. A pomalu se začíná vyjevovat pravda o tajném projektu z éry Kulturní revoluce.}}
\end{center}

\end{document}
