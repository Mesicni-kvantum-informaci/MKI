\documentclass{../../../../style/mkimain}

\series{3}
\month{duben}
\year{2023}

\begin{document}
\ExecuteMetaData[../../problems/problemT3-U2/problemT3-U2.tex]{header}
\noindent\ExecuteMetaData[../../problems/problemT3-U2/problemT3-U2.tex]{task}
\proborigin{Michal a Vojta}
\klein
Pojďme si postupně projít všechny možnosti a zamyslet se nad správným řešením. Jak vám už asi došlo
tato úloha má 3 správná řešení a), b) a c). 

První možnost je podle klasické matematiky \uv{nejkorektnější}. 
S každým číslem se součet zvětšuje, takže intuitivně dává smysl, že součet nekonečného počtu čísel bude právě nekonečno. 
Tato velice intuitivní myšlenka naštěsí pro tento konkrétní případ platí (složitější matematikou lze dokázat, že daná číselná řada 
\emph{diverguje} tj. součet všech jejích členů je $\infty$). Avšak je dobré pamatovat si, že ne všchny sumace jsou takto intuitivní.
Například sumace převrácených hodnot mocnin čísla 2 \emph{konverguje}\footnote{Pokud se chcete dozvědět více o konvergentních a 
divergentních řadách: \newline \url!https://en.wikipedia.org/wiki/Convergent_series#Examples_of_convergent_and_divergent_series!} k výsledku 2.
$$
\sum_{n=0}^{\infty}\frac{1}{2^n}=\frac{1}{2^0}+\frac{1}{2^1}+\frac{1}{2^2}+\frac{1}{2^3}+\frac{1}{2^4}+\frac{1}{2^5}+\dots=
\frac{1}{1}+\frac{1}{2}+\frac{1}{4}+\frac{1}{8}+\frac{1}{16}+\frac{1}{32}+\dots=2
$$

Teď se podíváme na možnost c). 
\end{document}
