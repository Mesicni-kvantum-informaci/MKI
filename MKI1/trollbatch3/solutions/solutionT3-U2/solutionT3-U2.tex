\documentclass{../../../../style/mkimain}

\series{3}
\month{duben}
\year{2023}

\begin{document}
\ExecuteMetaData[../../problems/problemT3-U2/problemT3-U2.tex]{header}
\noindent\ExecuteMetaData[../../problems/problemT3-U2/problemT3-U2.tex]{task}
\proborigin{Michal a Vojta}
\klein
Pojďme si postupně projít všechny možnosti a zamyslet se nad správným řešením. Jak vám už asi došlo
tato úloha má 3 správná řešení a), b) a c). 

První možnost je podle klasické matematiky \uv{nejkorektnější}. 
S každým číslem se součet zvětšuje, takže intuitivně dává smysl, že součet nekonečného počtu čísel bude právě nekonečno. 
Tato velice intuitivní myšlenka naštěsí pro tento konkrétní případ platí (složitější matematikou lze dokázat, že daná číselná řada 
\emph{diverguje} tj. součet všech jejích členů je $\infty$). Avšak je dobré pamatovat si, že ne všchny sumace jsou takto intuitivní.
Například sumace převrácených hodnot mocnin čísla 2 \emph{konverguje}\footnote{Pokud se chcete dozvědět více o konvergentních a 
divergentních řadách: \newline \url!https://en.wikipedia.org/wiki/Convergent_series#Examples_of_convergent_and_divergent_series!} k výsledku 2.
$$
\sum_{n=0}^{\infty}\frac{1}{2^n}=\frac{1}{2^0}+\frac{1}{2^1}+\frac{1}{2^2}+\frac{1}{2^3}+\frac{1}{2^4}+\frac{1}{2^5}+\dots=
\frac{1}{1}+\frac{1}{2}+\frac{1}{4}+\frac{1}{8}+\frac{1}{16}+\frac{1}{32}+\dots=2
$$

\noindent Zdůvodnění správnosti druhé možnosti je celkem jasné. 42. Odpověď na všechno.

Někteří z vás se určitě nedočkavě ptají jak je to s možností c). I přes veškerou intuici je 
i tato možnost správná. Existuje několik způsobů jak sumaci (součtu) všech přirozených čísel 
přiřadit\footnote{Záměrně jsme použili slovo přiřadit, protože následující hodnota není výsledkem sumace ve klasickém smyslu} 
hodnotu $-\dfrac{1}{12}$, nejznámějšími jsou tzv. \emph{regularizace zeta funkce} a \emph{Ramanujanova sumace}. 
\\
\\
\noindent \emph{Upozornění pro nematematiky: Obě metody vyžadují využití složité matematiky (calculus, komplexní čísla). Níže se vám pokusíme metody vysvětlit ve zjednodušené formě.}

\subsection*{Regularizace zeta funkce}
Náš příběh začíná u slavného matematika \emph{Leonharda Eulera} a jiné nekonečné řady, a to u sumace všech mocnin daného čísla $x$.
$$
\sum_{n=0}^{\infty}x^n=1+x+x^2+x^3+x^4+x^5+\dots=\frac{1}{1-x}
$$
Nyní si stějně jako Euler dokážeme výše uvedenou hodnotu, ke které daná řada konverguje. Označme tuto řadu $S$.
$$
S=1+x+x^2+x^3+x^4+x^5+\dots
$$
Po vynásobení obou stran rovnice $x$ dostáváme 
$$
xS=x+x^2+x^3+x^4+x^5+x^6+\dots\text{.}
$$
Když od sebe tyto dvě řady odečteme všechny jejich členy se až na 1 z první řady vykrátí.
$$
S-xS=1
$$
Několika jednoduchými úpravami z této rovnice vyjádříme $S$.
$$
S(1-x)=1
$$
$$
S=\frac{1}{1-x}
$$
Nyní, když už známe hodnotu této sumace, se pojdmě vrátit zpět k úpravám. 
K dalšímu postupu budeme řadu potřebovat \emph{zderivovat}. K derivování 
této řady naštěstí potřebujeme znát jen jedno jednoduché pravidlo, a tím je pravidlo pro derivaci mocniny.
$$
\dv{x}x^n=nx^{n-1}
$$
Teď zderivujeme obě strany rovnice. Derivace 1 je 0. Pravou stranu rovnice můžeme zapsat 
jako $\left(1-x\right)^{-1}$ a uplatnit pravidlo zmíněné výše.
$$
\dv{x}\left(1+x+x^2+x^3+x^4+x^5+x^6+\dots\right)=\dv{x}\left(\frac{1}{1-x}\right)
$$
$$
0+1+2x+3x^2+4x^3+5x^4+6x^5+\dots=\frac{1}{\left(1-x\right)^2}
$$
V dalším kroku Euler položil $x=-1$. Po dosazení této hodnoty dostaneme následující rovnici.
$$
1-2+3-4+5-6+\dots=\frac{1}{4}
$$
Teď už přichází do hry funkce zmíněná v názvu této metody, \emph{Riemannova zeta funkce}
\footnote{Můžete se setkat i s označením \emph{Eulerova-Riemannova zeta funkce}, Euler jí formuloval a Riemann jí rozšířil pro obor komplexních čísel.}.
Značí se $\zeta(s)$, kde $s=\sigma+ti \in \mathbb{C}$ je její argument náležící \emph{komplexním číslům} s \emph{reálnou} částí $\operatorname{Re}(s)=\sigma$ 
a \emph{imaginární} částí $\operatorname{Im}(s)=t$, a je definování následovně:
$$
\zeta(s)=\sum_{n=1}^{\infty}\frac{1}{n^s}=\frac{1}{1^s}+\frac{1}{2^s}+\frac{1}{3^s}+\frac{1}{4^s}+\frac{1}{5^s}+\dots=1^{-s}+2^{-s}+3^{-s}+4^{-s}+5^{-s}+\dots
$$
Když $\zeta(s)$ vynásobíme $2^{-s}$ dostaneme
$$
2^{-s}\zeta(s)=2^{-s}+4^{-s}+6^{-s}+8^{-s}+10^{-s}+\dots
$$
Teď dvojnásobek tohoto výsledku odečteme od zeta funkce $\zeta(s)$.
$$
\zeta(s)-2\cdot2^{-s}\zeta(s)=(1-2\cdot2^{-s})\zeta(s)=1-2^{-s}+3^{-s}-4^{-s}+5^{-s}-6^{-s}+\dots
$$
Dále Euler položil $s=-1$, takže naštěstí nebudeme muset počítat s komplexními čísly. Po dosazení do obou stran rovnice dostaneme:
$$
(1-2\cdot2^1)\zeta(-1)1-2^{1}+3^{1}-4^{1}+5^{1}-6^{1}+\dots
$$
Po dosazení $-1$ do argumentu $\zeta(s)$ se nám celá funkce redukuje na součet všech přirozených čísel. To zní povědomě, ne?
$$
-3(1+2+3+4+5+\dots)=1-2+3-4+5-6+\dots
$$
Už dříve jsme dokázali, že pravá strana rovnice je rovna $\dfrac{1}{4}$, takže už nás čeká jen pár úprav.
$$
-3(1+2+3+4+5+\dots)=\frac{1}{4}
$$
$$
1+2+3+4+5+\dots=-\frac{1}{12}
$$
A máme výsledek!
\subsection*{Ramanujanova sumace}
Tato metoda je o něco kratší než předchozí, ale neméně zajímavá. V hodně zkrácené a zjednoduššené verzi zní takto:
Ozančme si sumaci všech přirozených čísel $c$.
$$
c=1+2+3+4+5+6+\dots
$$
Teď zapišme čtřnásobek této řady $4c$.
$$
4c=4+8+12+16+20+24+\dots
$$
Když teď od sebe tyto dvě řady odečteme dostaneme posloupnost přirozených čísel s \uv{prohozeným} každým druhým znaménkem.
$$
-3c=1-2+3-4+5-6+\dots
$$
\emph{Srinivasa Ramanujan} dále ve svém postupu dokázal stejně jako my při řešení přes zeta funkci, že tato posloupnost je rovna $\dfrac{1}{4}$. 
$$
-3c=\frac{1}{4} \ \rightarrow \ c=-\frac{1}{12}
$$
$$
1+2+3+4+5+\dots=-\frac{1}{12}
$$
Sice jsme se podobným (a jednodušším!) způsobeb dostali ke stejnému výsledku jako 
Euler při odvození přes zeta funkci, ale tento postup není úplně matematicky korektní.
Určitě jste si všimli, že když jsme od původní řady odečítali její čtyřnásobek, aby se 
nám \uv{otočilo} znaménko každého druhého členu. To s sebou ale nese menší problém. 

A to sice že trochu zapomínáme, že manipulujeme s nekonečnými řadami, a že s nimi nemůžeme 
zacházet stejně jako s konečnými. Ve skutečnosti jsme totiž odčítali řadu
$$
0+4+0+8+0+12+0+\dots\text{,}
$$
což by se na první pohled mohlo zdát jen jako nevinná úprava pro přehlednost, ale při práci 
s nekonečmi řadami si musíme dávat pozor i na takové věci. Přidáním jedné nuly např. na začátek bychom mohli změnit 
celý výsledek. A nejen to! Řada by pak dokonce ani nemusela konvergovat! 

%Druhým problémem je velikost řad. Možná si říkáte, že obě řady jsou přeci nekonečné, tak v čem je problém?
%Háček spočívá v tom, že ne všechna nekonečna jsou stejně velká. Podívejme se například na přirozená a reálná čísla. 
%Představte si vypíšete pod sebe všechna reálná čísla a ke každému přiřadíte jedno přirozené číslo. Teď máte na \uv{papíře} 
%stejný počet čísel z obou množin a mohlo by se zdát, že obě množiny mají stejný počet čísel. Opak je ale pravdou. Pokaždé, 
%když si myslíte, že jste vypsali všechna reálná čísla, tak lze najít jedno, které jste ještě nevypsali.
%Jednosuše vezmete první číslici prvního čísla a změníte jí, pak druhou číslici druhého čísla atd. Takto dostanete nové číslo, 
%které se od každého čísla v seznamu liší minimálně jednou číslicí. Co s tím? Protože jsme všechna přirozená čísla už vyčerapali na předchozí reálná čísla musí reálných čísel být více než přirozených. 
%Pojdmě se vrátit zpět k řadám, které jsme odčítali. Abychom získali výslednou řadu tak jsme vždy odečetli jeden člen z řady $4c$ od každého druhého členu řady $c$. 
%Není to trochu problém? Odčítáme \uv{méně} čísel 

% jsem taky debil

\end{document}
