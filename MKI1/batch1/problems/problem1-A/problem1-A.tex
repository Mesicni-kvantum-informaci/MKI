\documentclass[crop=false]{standalone}
\usepackage{siunitx}
\sisetup{output-decimal-marker = {,}}

\begin{document}
\section*{Základní orientace na obloze}
% Historie
\quad

Lidé vzhlíží k hvězdné obloze a obdivují její krásu již od nepaměti.
Avšak nejen to.
Díky obloze se orientují na svých cestách či třeba vytváří kalendáře.

%Ekliptika, proč se obloha otáčí 
Začněmě tím, proč se vlastně obloha(neboli nebeská sféra) během roku mění.
Asi všem je jasné, že Země obíhá kolem Slunce a otáčí se kolem své osy.
Dráha, po které se Slunce pohybuje na obloze se nazývá \textit{ekliptika}. Je to průmět pohybu Země kolem Slunce na nebeskou sféru.
Jelikož všechny planety mají podobný sklon roviny oběhu kolem Slunce, najdeme kolem ekliptiky i planety.

%Světový rovník
Sklon rotační osy Země je zhruba \qty{23,5}{\degree}.
To znamená, že úhel který osa svírá s \textit{nebeským rovníkem} (průmět zemského rovníku na oblohu) je \qty{66,5}{\degree} . 
Kvůli oběhu Země kolem slunce se ekliptika a nebeský rovník spolu po obloze hýbají.
Proto je v zimě Slunce nízko, a v létě vysoko.

%Polárk
Dalším pojmem, který budem potřebovat je \textit{nebeský severní a jižní pól}.
Opomenu-li málo výrazné pohyby Země, které mají vliv na sklon rotační osy, míří severní pól stále k stejné hvězdě, \textit{Polárce} (Severka, Polaris, $\alpha$ Ursae Minoris).
To znamená, že Polárka bude na obloze vždy na "stejném místě".
Kde přesně?
Víme, že severní pól a rovník svírají úhel \qty{90}{\degree}. Proto budeme Polárku hledat \qty{90}{\degree} severně od nebeského rovníku. 
Odborně bychom řekli, že \textit{deklinace} Polárky je zhruba \qty{90}{\degree}. 


%88 souhvězdí, Severní a jižní polokou
%Velký a malý vůz -> cirkumpolární souhvězdí
%Jarní, letní, podzimní, zimní souhvězdí
%Ekliptikální(zvířetníková) souhvězdí
%Základy sférické astronomie
\end{document}

%Stellarium
