\documentclass{../../../../style/mkimain}

\series{1}
\month{únor}
\year{2023}

\begin{document}
%<*header>
\section*{I.A Základní orientace na obloze}
%</header>

% Historie

Lidé vzhlíží k hvězdné obloze a obdivují její krásu již od nepaměti.
Avšak nejen to.
Díky obloze se orientují na svých cestách či třeba vytváří kalendáře.

%Ekliptika, proč se obloha otáčí 
Začněmě tím, proč se vlastně obloha (nebeská sféra) během roku mění.
Asi všem je jasné, že Země obíhá kolem Slunce a otáčí se kolem své osy.
Dráha, po které se Slunce pohybuje na obloze se nazývá \textit{ekliptika}. Je to průmět pohybu Země kolem Slunce na nebeskou sféru.
Jelikož všechny planety mají podobný sklon roviny oběhu kolem Slunce, najdeme kolem ekliptiky i planety.

%Světový rovník
Sklon rotační osy Země je zhruba \qty{23,5}{\degree}.
To znamená, že úhel který osa svírá s \textit{nebeským rovníkem} (průmět zemského rovníku na oblohu) je \qty{66,5}{\degree} . 
Kvůli oběhu Země kolem slunce se ekliptika a nebeský rovník spolu po obloze hýbají.
Proto je v zimě Slunce nízko, a v létě vysoko.

%Polárk
Dalším pojmem, který budem potřebovat je \textit{nebeský severní a jižní pól}.
Opomenu-li málo výrazné pohyby Země, které mají vliv na sklon rotační osy, míří severní pól stále k stejné hvězdě, \textit{Polárce} (Severka, Polaris, $\alpha$ Ursae Minoris).
To znamená, že Polárka bude na obloze vždy na \uv{stejném místě}.
Kde přesně?
Víme, že severní pól a rovník svírají úhel \qty{90}{\degree}. Proto budeme Polárku hledat \qty{90}{\degree} severně od nebeského rovníku. 
Odborně bychom řekli, že \textit{deklinace} Polárky je zhruba \qty{90}{\degree}.
Důležitá je Polárka hlavně v tom, že se kolem ní \uv{otáčí} obloha. 

%88 souhvězdí, Severní a jižní polokou
Po pochopení proč a jak se obloha mění, se můžeme zabývat tím, co se na obloze nachází.
Nejvýraznějšími útvary na obloze jsou \textit{souhvězdí}.
Často si lidé milně domnívají, že se jedná pouze o obrazce tvořené jasnými hvězdami.
V moderní astronomii je však souhvězdí oblast na obloze s přesně vymezenými hranicemi.
Na nebi jich bylo přesně vymezeno 88. Většina souhvězdí  viditelných ze  severní polokoule převzalo název z dob antických.
Souhvězdí na jižní obloze mají názvy většinou od mořeplavců, kteří se vydávali na daleké výpravy.

%Velký a malý vůz -> cirkumpolární souhvězdí
Po celý rok na obloze najdete \textit{cirkumpolární souhvězdí}. 
Vídime je v jakémkoliv ročním období jelikož se pro pozorovatele na Zemi nacházejí na obloze blízko Polárky, tedy hvězdy kolem které se celá obloha otáčí.
Polárka je součástí souhvězdí Malý Medvěd.
Poblíž se nachází Velká medvědice, jejíž částí je všem známý Velký vůz.

%Jarní, letní, podzimní, zimní souhvězdí
Cirkumpolárních souhvězdí není mnoho, pouze 8. Všechna ostatní rozdělujeme podle toho, kdy jsou v noci vidět.
Tedy jarní, letní, podzimní a zimní souhvězdí.
Každá z obloh má svůj hlavní orientační obrazec, skládající se z nejjasnějších hvězd různých souhvězdí.
Na jarní obloze se budem orientovat pomocí Jarního trojúhelníku a v létě nám pomůže trojúhelník letní.
Na podzim si nelze nevšimnout výrazného Pegasova čtverce.
A konečně na zimní obloze, v pozadí s Mléčnou dráhou, je obrazec tvořený šesti velmi jasnými hvězdami, Zimní šestiuhelník.
Tyto obrazce nám pomáhají rychle se orientovat na obloze.
Pozorujete-li oblohu, nejprvě si najděte tento obrazec. Odtud lehce naleznete požadované souhvězdí.
Věřím, že není nutné vyjménovávat všechna souhvězdí\dots jednoduše si na internetu či v knihách najděte seznam sami.
Pro pozorování je vemi dobrou pomůckou otočná mapa, či dnes více popularní, mobilní aplikace.

%mlhave objekty...galaxie...
Na obloze nenajdeme jen hvězdy, planety, Měsíc a Slunce. Avšak záleží kde pozorujete, v městech nemusíte najít ani to.
Doopravdy je celá obloha poseta galaxiemi, mlhovinami, hvězdokupami a mnoho dalším. 
To si ale necháme na seriál o hlubokém vesmíru, ať se máte na co těšit.
\\
\\
\textbf{Úlohy:} % NÁZEV: ÚLOHY/OTÁZKY ??? ??? ???
\\
\\
%<*task>
V seriálu jsem psal o souhvězdích severní oblohy a jižní oblohy. Vysvětlete, co to je jižní a severní obloha, a proč nějaké souhvězdí přiřazujeme severní obloze a jiné jižní.
\\
\\
Jelikož nad hlavami právě máme zimní oblohu, pozorujte v noci Zimní šestiúhelník. Která planeta se momentálně nachází \uv{uvnitř} tohoto obrazce?
%</task>

\end{document}
