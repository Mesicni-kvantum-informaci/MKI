\documentclass[crop=false]{standalone}
\begin{document}
\section*{Jak je to asi pravděpodobné?}
\quad

Co je to \textit{kvantový svět}? Tento pojem můžete slyšet v médiích v souvislosti s moderní fyzikou zcela běžně. Víte ale, co přesně tento pojem znamená? Místo toho, abychom si ukazovali nějakou nicneříkající školní definici, se pokusme zamyslet nad tím, jak se vlastně takový svět projevuje\dots
\\

V našem každodenním životě jsme zvyklí na klasickou fyziku, kde můžeme přesně určit a naměřit všechny možné fyzikální veličiny. Běžně říkáme, že si například dáme sraz u KFC, že aktuálně jedeme autem rychlostí $60\;\mathrm{km\cdot h^{-1}}$, a podobně. Žijeme ve světě, ve kterém jsou čas, poloha, rychlost a podobné veličiny naprosto konkrétní a určité. Napadlo vás ale někdy, že my lidé jsme jakožto měřící přístroje zcela nepřesní? Dvě události dokážeme rozlišovat jenom tehdy, když proběhnou alespoň $0,02\;\mathrm{s}$ po sobě, nerozeznáme věci menší než pár mikrometrů, naše pozorování jsou tedy velmi omezená. Je tedy samozřejmé, že nám lidem se může zdát, že vidíme ku příkladu umístění objektů zcela přesně, jenže zároveň pro nás jsou délky jako velikost mitochondrie či vlnová délka viditelného světla
naprosto zanedbatelné, přitom právě na této škále se dějí ty největší zázraky přírody. 

Veličiny jako rychlost nebo poloha nejsou vůbec konkrétní a nedá se přesně říct, jakých hodnot nabývají. Dá se ovšem spočítat, s jakou pravděpodobností bude částice takovou hybnost nebo polohu mít. A právě zde začínají nejzákladnější principy kvantového světa. Kvantový svět není určitý, je to svět pravděpodobností. Naprosto cokoliv se zde může stát, i když se to příčí klasické newtonovsvké mechanice, nicméně všechno je omezeno určitou pravděpodobností.

Je kupříkladu možné, že během čtení tohoto seriálu od vás odletí jeden elektron, vystartuje ze Země k Proximě Centauri, oběhne ji a vrátí se zpátky do vašeho těla až odmaturujete. Takový proces je zcela validní (i když dle klasické fyziky by na to elektron neměl ani zdaleka dost energie), ale vysoce nepravděpodobný, že je až šílenost věřit, že se to komukoliv v historii lidstva povedlo.

Možné je rovněž to, že se nějaký učitel biologie během svého výkladu o ornitologii promění ve volavku popelavou a vytvoří tak nejlepší praktickou ukázku v historii učitelství. Pokud by se každý atom v těle a v okolí přeskupil na správné místo, může tato situace nastat.
\\

V tomto seriálu jsme zjistili, že možné je opravdu cokoliv. Část vašeho těla může být vyslána na první interstelární misi aniž byste o tom věděli, mezi učiteli se může skrývat potenciální zvěromág\dots Možností je vskutku nekonečně mnoho. Závěrem by nám mohlo být poznání toho rozdílu, že v klasickém světě se ptáme, zda se může něco stát, ovšem ve světě kvantovém je ta správná otázka: \textit{jak je to asi pravděpodobné?}

Jak se nazývá princip, který pojednává o nemožnosti přesného měření hybnosti (rychlosti) a polohy?
a) Robertsonův vztah
b) Pauliho vylučovací princip
c) Heisenbergova relace neurčitosti
d) Hundovo pravidlo

\end{document}
