\documentclass[crop=false]{standalone}
\begin{document}
\section*{I.B Uhlo-vodík}
Jakou rychlostí by se musel pohybovat atom vodíku, aby měl z pohledu nehybného pozorovatele stejnou hmotnost jako atom uhlíku v klidu? 
Výsledek vyjádřete v násobcích $c$ (rychlosti světla).
\end{document}
