\documentclass{../../../../style/mkimain}

\series{1}
\month{únor}
\year{2023}

\begin{document}
%<*header>
\section*{I.B Uhlo-vodík}
%</header>
%<*task>
\noindent Jakou rychlostí by se musel pohybovat atom vodíku, aby měl z pohledu nehybného pozorovatele stejnou hmotnost jako atom uhlíku v klidu? Výsledek vyjádřete v násobcích $c$ (rychlosti světla).
%</task>
\end{document}
