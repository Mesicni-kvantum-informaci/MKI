\documentclass{../../../../style/mkimain}

\series{1}
\month{únor}
\year{2023}

\begin{document}
\ExecuteMetaData[../../problems/problem1-U3/problem1-U3.tex]{header}
\noindent\ExecuteMetaData[../../problems/problem1-U3/problem1-U3.tex]{task}
\proborigin{Michal chtěl zapálit svůj test z dějepisu a předstírat, že to byla nehoda.}
\klein
Schopnost směrovat paprsky do jednoho ohniska, pokud jdou rovnoběžně s optickou osou, má pouze \textbf{konkávní parabolické zrcadlo}.
Slovo \uv{konkávní} jednoduše odkazuje na stranu, na kterou paprsky dopadají, hlavní však je určit přesný tvar zrcadla.
Přesně se tento tvar musí počítat buďto geometricky podle zákona odrazu, nebo pomocí tzv. \emph{Fermatova principu nejkratšího času},
ze kterého pak v našem případě vyplývá jeden důležitý fakt.
Pokud bychom pustili libovolné množství světelných paprsků z roviny kolmé na optickou osu směrem do zrcadla tak,
aby letěly rovnoběžně s touto osou, pak platí, že se všechny tyto paprsky střetnou v ohnisku ve stejnou chvíli.
Matematicky to pak znamená, že urazí stejnou vzdálenost. Jediný objekt, který tento požadavek splňuje, je \emph{rotační paraboloid}.

Často se můžete doslechnout, že stejnou schopnost má i kulové zrcadlo. Není tomu úplně tak, platí to pouze přibližně,
pokud se paprsky pohybují blízko optické osy (zdatní matematici si tento fakt mohou dokázat třeba tzv.
\emph{limitou} či \emph{Taylorovým polynomem}).
Oblast v blízkosti optické osy, kde má kulové zrcadlo téměř stejné zobrazovací účinky jako parabolické, se nazývá \emph{paraxiální prostor}.

% formulace toho textu v závorce je dost divná (chápu, co se snažíš říct, ale napsat to nějak líp, nebo to vynechat) 
% also v těch vzorácích hodně flexíš s advanced matematickou (limitama, derivacema / taylorem), což nevim jestli je dobře - diskuze! -- Vojta

Výše popisovaného efektu se dá využít dobře i v praxi.
Příkladem mohou být sluneční ohřívače vody, které všechnu světelnou energii dopadající na zrcadlo koncentrují do malé konvice,
rovněž také běžné satelity nebo třeba legendární Archimédova soustava zrcadel, která měla údajně sloužit k zapalování nepřátelských lodí\dots 

\end{document}
