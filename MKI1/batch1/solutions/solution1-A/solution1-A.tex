\documentclass{../../../../style/mkimain}

\series{1}
\month{únor}
\year{2023}

\begin{document}
\ExecuteMetaData[../../problems/problem1-A/problem1-A.tex]{header}
\noindent\ExecuteMetaData[../../problems/problem1-A/problem1-A.tex]{task}
\proborigin{Jindra se při nočním běhání ztratil v lese.}
\klein
Severní obloha je ta část oblohy, kterou můžeme vidět ze severní polokoule Země. 
Stejně tak, jižní oblohu lze vidět z jižní polokoule Země.
Celou severní oblohu z jižní polokoule (a samozřejmě i naopak) vidět nemůžeme, jelikož je doslova zakrytá Zemí.
To ale neznamená, že ji nevidíme vůbec.
Například, pokud bychom byli na rovníku, viděli bychom polovinu severní, a polovinu jižní oblohy.
Podle výrazného rudého zbarvení lze lehce poznat, že planetou v zimním šestiúhelníku byl Mars. 
\end{document}
