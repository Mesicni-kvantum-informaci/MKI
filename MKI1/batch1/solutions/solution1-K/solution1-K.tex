\documentclass{../../../../style/mkimain}

\series{1}
\month{únor}
\year{2023}

\begin{document}
\ExecuteMetaData[../../problems/problem1-K/problem1-K.tex]{header}
\noindent\ExecuteMetaData[../../problems/problem1-K/problem1-K.tex]{task}
\proborigin{Michal přemýšlel nad pravděpodobností, že dostane jedničku z dějepisu}
\klein
Pojem \emph{Heisenbergův princip (relace) neurčitosti} je velmi dobře znám i laické veřejnosti.
Pojednává o nepřímé úměrnosti nepřesnosti měření polohy a hybnosti, jinými slovy čím přesněji určíme polohu částice,
tím méně přesně už můžeme určit její hybnost (samozřejmě i naopak). V jednodimenzionálním případě vypadá jeho matematická formulace následovně:
$$\Delta p \Delta x \geq \frac{\hbar}{2}\text{,}$$
kde $\Delta p$ a $\Delta x$ jsou nejistoty hybnosti a polohy a $\hbar$ značí tzv. \emph{redukovanou Planckovu konstantu}.

Identitou, která tuto neurčitost popisuje, může být i tzv. \emph{Robertsonův vztah},
který ale slouží v podstatě univerzálně a lze jím popsat relace neurčitosti mezi libovolnými veličinami popisujícími danou částici či celý systém.
Jedná se o takové zobecnění Heisenbergova principu na všechny možné veličiny.

\end{document}
