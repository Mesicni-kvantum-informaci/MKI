\documentclass{../../../../style/mkimain}

\series{1}
\month{únor}
\year{2023}

\begin{document}
\section*{I.B Uhlo-vodík}
\noindent Jakou rychlostí by se musel pohybovat atom vodíku, aby měl z pohledu nehybného pozorovatele stejnou hmotnost jako atom uhlíku v klidu? 
Výsledek vyjádřete v násobcích $c$ (rychlosti světla).
\klein
Jelikož se atom vodíku bude pohybovat rychlostí blízkou rychlosti světla musíme přestat 
uvažovat o jeho hmotnosti jako o konstantě. Vztah mezi \emph{relativistickou hmotností} $m$ 
a \emph{klidovou hmotností} $m_0$ je dán následujícím vzorcem.
$$
m=m_0\gamma\text{,}
$$
kde $\gamma=\left(1-\frac{v^2}{c^2}\right)^{-\frac{1}{2}}$ je Lorentzův faktor.

Klidovou hmotnost atomu vodíku označíme $m_\mathrm{H}$ a jeho relativistickou hmotnost, která 
bude rovna hmotnosti atomu uhlíku, označíme $m_\mathrm{C}$. 
$$
m_\mathrm{C}=\frac{m_\mathrm{H}}{\sqrt{1-\frac{v^2}{c^2}}}
$$
Několika úpravami vyjádříme rychlost $v$.
$$
\sqrt{1-\frac{v^2}{c^2}}=\frac{m_\mathrm{H}}{m_\mathrm{C}}
$$
$$
\frac{v^2}{c^2}=1-\left(\frac{m_\mathrm{H}}{m_\mathrm{C}}\right)^2
$$
$$
v^2=\left(1-\left(\frac{m_\mathrm{H}}{m_\mathrm{C}}\right)^2\right)c^2
$$
$$
v=c\sqrt{1-\left(\frac{m_\mathrm{H}}{m_\mathrm{C}}\right)^2}
$$
Za $m_\mathrm{H}$ a $m_\mathrm{C}$ dosadíme relativní atomové hmotnosti.
$$
m_\mathrm{H}=\mathrm{A_r(\ce{H})}=1,008
$$
$$
m_\mathrm{C}=\mathrm{A_r(\ce{C})}=12,011
$$
$$
v=c\sqrt{1-\left(\frac
{1,008}{12,011}\right)^2}\approx0,996\ c
$$
Aby atom vodíku měl stejnou hmotnost jako atom uhlíku v klidu musel by se pohybovat rychlostí cca $0,996\ c$.\\
\end{document}
