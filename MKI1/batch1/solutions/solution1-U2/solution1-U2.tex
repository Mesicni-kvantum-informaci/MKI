\documentclass{../../../../style/mkimain}

\series{1}
\month{únor}
\year{2023}

\begin{document}
\ExecuteMetaData[../../problems/problem1-U2/problem1-U2.tex]{header}
\noindent\ExecuteMetaData[../../problems/problem1-U2/problem1-U2.tex]{task}
\proborigin{Jindra se zase díval na Rande s Fyzikou.}
\klein
Zvolme soustavu s počátkem ve středu Země. První osa prochází ISS, druhá je na ní kolmá. 
Jelikož se ISS pohybuje po kružnici (doopravdy po elipse, ale aproximujeme\dots), otáčí se i osa.
Jedná se tedy o neinerciální soustavu.
Působí na ní síla dostředivá~–~gravitační a síla setrvačná~–~odstředivá.
ISS se v této soustavě nepohybuje, a proto se tyto síly rovnají, tedy výslednice těchto sil je nulová, a astronauti se vznáší.


\end{document} 
