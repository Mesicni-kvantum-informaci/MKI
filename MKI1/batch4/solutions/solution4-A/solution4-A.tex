\documentclass{../../../../style/mkimain}

\series{4}
\month{květen}
\year{2023}

\begin{document}
\ExecuteMetaData[../../problems/problem4-A/problem4-A.tex]{header}
\noindent\ExecuteMetaData[../../problems/problem4-A/problem4-A.tex]{task}
\proborigin{Jindra se na svůj prospěch dívá raději logaritmickou škálou.}
\klein
\begin{enumerate}
\item
Hvězdy vyzařují podle Planckova zákona v celém barevném spektru, ovšem někde méně, někde více.
Právě podle dominantní vlnové délky se většinou řídí barva hvězdy, jenže ne vždy.
Lidské oko nevidí bohužel celé spektrum, vidí jen nějakou část, tzv. viditelné světlo.
A z této části si samo aditivně skládá barvy zhruba podle poměru intezit, ve kterých je vidí.
Lidský mozek si během evoluce vymyslel vlastní barvu, kterou vnímá, vidí-li oko viditelné spektrum přibližně rovnoměrně.
Tou barvou je bílá. Aby byla závislost intezity na vlnové délce v pouhém viditelném spektru přibližně konstantní,
musí být vrchol křivky hezky symetricky uprostřed tohoto intervalu, to znamená někde kolem zelené. A to je důvod, proč nevidíme zelené hvězdy,
naše oko a náš mozek si jejich spektrum složí do něčeho, čemu říkáme bílá barva.
\\
\\
Proč ale nevidíme fiolovou? Je přece na konci viditelného spektra. Ano, právě proto, tahle barva je moc na hranici,
takže ji silně přebarvují i slabé vlivy ostatních barev. Navíc je biologicky dokázáno, že lidské oko vnímá mnohem lépe modrou nežli fialovou.
Vzhledem k tomu, že hvězdy s fialovým spektrem vyzařují rovněž hodně na modrých vlnových délkách, je potom jasné, že naše oko zaregistruje lépe právě modrou.
\item Vnímání oka je \emph{logaritmické}. To je klíčový pojem pro vizuální a dokonce i sluchový vjem. Intenzita záření jako veličina slábne s druhou mocninou vzdálenosti.
To znamená, že když se kupříkladu dvakrát vzdálíme od bodového zdroje, intenzita zeslábne čtyřikrát. Jenže to se nám lidem nezdá, že?
Vidíme podobně jasně lampu, která je 50 metrů od nás a lampu, která je od nás 100 metrů. To právě díky tomu, že naše oko vnímá podle logaritmické funkce intenzity.
HR diagram se snaží lidem vyhovět i v jejich vnímání světa, a proto má jako jednu z os intenzitu v tzv. logaritmické škále, která je oproti klasické trochu zdeformovaná. Jenže pro člověka je to přirozenější.
\end{enumerate}
\end{document}
