\documentclass{../../../../style/mkimain}

\series{4}
\month{květen}
\year{2023}

\begin{document}
\ExecuteMetaData[../../problems/problem4-U2/problem4-U2.tex]{header}
\noindent\ExecuteMetaData[../../problems/problem4-U2/problem4-U2.tex]{task}
\proborigin{Kdyby Michal narušil kauzalitu času, podíval by se, jestli získá Nobelovku.}
\klein
Kvantově provázat můžeme opravdu velkou škálu veličin různých systémů.
Z hlediska kvantové komunikace jsou pravděpodobně nejatraktivnější libovolné diskrétní veličny,
jako je třeba spin elektronu nebo polarizace fotonu. Dva nejznámější příklady.
Vezměme si tedy jako ukázku dva elektrony s provázanými spiny. Separujeme je daleko od sebe, tisíce světelných let.
Co si skrz ně můžeme poslat za infromaci? Aby ten druhý byl schopen informaci z elektronu vykoukat,
musí se na elektron podívat, provést měření. Měřením ale vlnová funkce zkolabuje do jasného stavu,
kdežto předtím byly oba elekrony v superpozici obou spinů, jednoduše měli 50\% pravděpodobnost, že mají spin nahoru
a 50\% pravděpodobnost, že spin dolů. V čem je problém? Přijímání informace je jen věc náhody. Držitel prvního elektronu (vysílače)
nemá šanci ovlivnit stav svého elektronu ani druhého. Dokonce ani sám neví, jaký spin má jeho elektron, a nezjistí to, dokud se nepodívá.
Když se ale podívá, provede měření, ovlivní tak stav prvního i druhého elektronu a vyšle jasnou zprávu svému kolegovi.
Oba budou najednou vědět, co má ten druhý u sebe za stav. Jenže ten stav bude naprosto náhodně určený a neřízený.
Nelze pak použít provázané částice jako systém nadsvětelně rychlé komunikace.
\\\\
Mnoho lidí se snaží vymyslet nějaký protiargument k tomuto tvrzení. Jeden z nich může být například ten,
že spin my jsme schopni laboratorně ovlivnit tak, aby se z 50\% pravděpodobnosti, že je spin nahoru, stala 100\%.
Například po vložení částice do magnetického pole, to pak dochází k tzv. precesi spinu, kdy pravděpodobnost toho,
že má částice spin nahoru osciluje a kolísá. Někdo si může říct, proč nevychytat ten moment, kdy je na 100 \% jasné, že spin směřuje nahoru.
V tom momentu by se provedl kolaps a byla by poslaná řízená kvantová zpráva (při statistickém zapojení vícero provázaných částic by to mohlo jít). Má to háček. Výpočetně se ukazuje,
že po vložení částice do magnetického pole, může přavděpodobnost toho, že spin směřuje nahoru, začít stoupat i klesat 
a to právě zcela náhodně. Principu náhody se člověk jen tak nezbaví. Smysluplná komunikace je tedy vyloučená.
\end{document}
