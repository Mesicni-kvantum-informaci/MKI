\documentclass{../../../../style/mkimain}

\series{4}
\month{květen}
\year{2023}

\begin{document}
\ExecuteMetaData[../../problems/problem4-B/problem4-B.tex]{header}
\noindent\ExecuteMetaData[../../problems/problem4-B/problem4-B.tex]{task}
\proborigin{Jindra si rád hlídá svou hmotnost kreativními způsoby.}
\klein
Pro inspiraci uvádíme pár nápadů, jak jinak se dá hmotnost sebe sama měřit.
\begin{itemize}
    \item Ponořit se do nádrže s vodou, aby člověk volně ploval. Změřit výtlakem vody objem ponořené části,
    poté objem vynořené části. Pomocí Archimédova zákona zjistit hustotu a celkový objem, poté už jen dopočíst hmotnost.
    \item Využít kolegu, jehož hmotnost je známá, posadit ho na rovnoramennou houpačku do určité vzdálenosti a
    poté zkoušet posadit sebe a hledat v jaké vzdálenosti od středu otáčení dojde k rovnováze.
    \item Zavěsit kolegu na sportovní gumu, změřit protáhnutí u něj a pak u sebe. Jednoduchý siloměr.
    \item Využít Boyle–Mariottova zákona. Postavit kolegu na bosu a postavit sebe sama na bosu. Změřit propadnutí na bose.
    \item Posadit kolegu do člunu a posadit sebe sama do člunu. Odstrčit se navzájem uprostřed na vodě, měřit rychlost obou člunů a využít zákona zachování hybnosti. 
\end{itemize}
\end{document}
