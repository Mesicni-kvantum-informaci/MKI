\documentclass{../../../../style/mkimain}

\series{4}
\month{květen}
\year{2023}

\begin{document}
\ExecuteMetaData[../../problems/problem4-U3/problem4-U3.tex]{header}
\noindent
\proborigin{Michal zase dělá na stáži MKi.}
\klein
\begin{enumerate}
\item Pomocí vylučovací metody můžeme vyřadit molekulu fluoru,
    která pro svůj nízký počet atomů má pouze jeden tzv. \emph{stupeň volnosti}.
    Ten udává počet frekvencí, na kterých může molekula ramanovsky vyzařovat. Vzhledem k tomu, že v grafu jsou
    hned dva vrcholy, molekula vyzařuje alespoň na dvou frekvencích. Fluor to tedy být nemůže.
    Zbytek už je jen potřeba dohledat na internetu. Jedná se o metan.
\item I jádra se mohou hýbat. V molekule jádra kmitají běžně (říkáme, že vibrují). Jejich energie je kvantovaná podobně jako v harmonickém oscilátoru.
Při přechodu z jednoho vibračního stavu do druhého se uvolní energie v podobě ramanovského typu záření.
\end{enumerate}
\end{document}
