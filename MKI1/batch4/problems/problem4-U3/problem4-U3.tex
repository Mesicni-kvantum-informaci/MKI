\documentclass{../../../../style/mkimain}

\series{4}
\month{květen}
\year{2023}

\begin{document}
\pgfkeys{/pgf/number format/.cd,1000 sep={\,}}
%<*header>
\section*{IV.U3 Molekuly, molekuly, hýbejte se!}
%</header>
%<*task>

\noindent
\begin{enumerate}
    \item Které molekule pravděpodobně patří níže zobrazené Ramanovo spektrum?
    \begin{choices}
        \choice \ce{H_2O}
        \choice \ce{CO_2}
        \choice \ce{CH_4}
        \choice \ce{F_2}
      \end{choices}
      \begin{tikzpicture}
        \begin{axis}[
            width=17cm,
            height=9cm,
            xmin=700,
            xmax=3800,
            ymin=0,
            ymax=50,
            title={\large Ramanovo spektrum molekuly (teorie)},
            xlabel={\normalsize vlnočet [\si{\per\cm}]},
            ylabel={\normalsize intenzita},
            xtick={1000,1500,2000,2500,3000,3500},
            ymajorticks=false,
            minor xtick={700,800,900,1100,1200,1300,1400,1600,1700,1800,1900,2100,2200,2300,2400,2600,2700,2800,2900,3100,3200,3300,3400,3600,3700,3800},
            xminorgrids=false
        ]
        \addplot+[
            color=blue,
            mark=none,
            ]
        table[meta=ma]
        {data.txt};
        \end{axis}
        \end{tikzpicture}
      \item Co v molekule určuje toto (čistě Ramanovo) spektrum?
      \begin{choices}
        \choice Energetické hladiny přeskakujících elektronů
        \choice Energetické hladiny kmitajících jader
      \end{choices}
\end{enumerate}
%</task>
\end{document}
