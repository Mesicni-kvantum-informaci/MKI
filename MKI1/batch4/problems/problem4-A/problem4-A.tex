\documentclass{../../../../style/mkimain}

\series{4}
\month{květen}
\year{2023}

\begin{document}
%<*header>
\section*{IV.A Header4-A}
%</header>
"umelecky" uvod...

Hned na začátek zapátrejte v paměti, a vzpomeňte si na minulý seriál (II.K).
V tomto seriílu jste se dozvěděli o \textit{Planckově vyzařovacím zákonu}.
Víte tedy, že každé těleso vyzařuje na všech vlnových délkách, avšak na jedné nejvíce.
Právě tuto vlnovou délku vidíme.
Podle \textit{Wienova posunovacího zákonu} je tato vlnová délka nepřímo úměrná teplotě tělesa. 
To znamená, že čím vyšší teplota hvězdy je, tím více se maximání vlnová délka posouvá k modré části spektra.
Modré hvězdy jsou tedy teplejší než hvězdy červené.
%spektrum cerneho telesa

Při pohledu na hvězdnou oblohu si lze velmi snadno všimnout toho, že hvězdy jsou různě jasné. (----pridat poznamku pod carou o problem v cestine s jasnosti, zarivosti a svitivosti).
Jak ??jasně(jasnou)?? hvězdu vidíme závisí na \textit{hustotě zářivého toku} (neboli jasnosti) a vzdálenosti hvězdy.
Prozatím vzdálenost odložíme, a budeme počítat s tím, že pozorujeme hvězdy ve stejné vzdálenosti. 

Hustota zářivého toku (???dále jen h.z.t.???) je veličina popisující tok záření, který projde $1\;\mathrm{m^2}$ za $1\;\mathrm{s}$.
H.z.t. dává do vztahu společne s teplotou tělesa \textit{Stefan-Boltzmanův zákon}. (---o odmitnutio sklonovani i prvniho jmena---)
Ten říka, že h.z.t. je přímo úměrná čtvrté mocnině teplotě tělesa.
Proto, čím teplejší hvězda je, tím více energie vyzařuje – má větší jasnost,??je jasnější??.

Jako míru jasnosti používáme hvězdnou velkost, neboli \textit{magnitudu}.
Tato míra odpovídá historickému dělení hvězd do šesti skupin, kde 0 byla nejjasnější a 5 nejméně jasná.
Dnes však magnutudu používáme pro všechny objekty na obloze, proto může jít i do záporu.
Například, nejjasnější objekt na obloze – samozřejmě Slunce – má magnitudu $-26,6$.
Chceme-li však porovnávat magnitudu nezávisle na vzdálenosti objektu, používáme \textit{absolutní magnitudu}. 
Jedná se magnitudu tělesa ve vzdálenosti $10\;\mathrm{pc}$ od nás. 

Různě veliké hvězdy mohou být stejně jasné (mít stejnou h.z.t.), avšak budou různě zářivé.
??(Opět, je potřeba si ujasnit co je to jasnost a co zářivost/svítivost).??
Proto zavádíme další veličinu, která nám popisuje jak hvězdu vidíme, a to \textit{zářivý výkon}.
(Nedborně bychom mohli říct zářivost či svítivost).
Jak je ale možné, že dvě stejně jasné hvezdy uvidíme jinak zářivé?
Odpověd spočívá v odlišných rozměrech.  
Hvězdy vyzařují energii svým povrchem.
Je-li jedna hvězda o stejné jasnosti vetší než druhá, má větší povrch, bude zářit více.
Zkrátka, má vetší plochu ze které září.
Není tedy težké domyslet, že zářivý výkon vypočítáme tak, že h.z.t. vynásobíme povrchem hvězdy.

Jak jsem již na samém začátku avizoval, hvězdy mají různé barvy.
???To znamená, že světlo které k nám od nich příchází, má různe spektrum.???
Na základě toho byly vytvořeny tzv. \textit{spektrální třídy}. (---poznamka o vtipnych mnemotechnićkych pomuckach pro zapomatovani si---)
Původně byly hvězdy rozděleny do sedmi skupin, kde každá skupina má deset podskupin.
S rozvojem techniky rozsah nestačil, a tak se třídy postupně rozrostly do dnešních třinácti.


hvezdna velikost
spektralni tridy
HR digram – spojeni vseho dohormady

proc jsou hvezdy koule
proc jsou tak horke - na cem zavisi teplota

otazky:
Proč nevidíme zelené nebo fialové hvězdy?

%\\
%\\
%\textbf{Úloha:}

%<*task>

%\begin{enumerate}[\noindent {}]
%\item otazka
%  \begin{choices}
%    \choice
%  \end{choices}
%\end{enumerate}

%</task>
\end{document}
