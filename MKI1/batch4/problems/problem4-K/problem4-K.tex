\documentclass{../../../../style/mkimain}

\series{4}
\month{květen}
\year{2023}

\begin{document}
%<*header>
\section*{IV.K Částice či vlna, to je oč tu běží}
%</header>
%úvod
V minulém seriálu jsme si odpověděli na otázku vyzařování a představili jsme si \emph{Planckův vyzařovací zákon}. 
Také jsme otevřeli téma tzv. \emph{částicově vlnového dualismu} (světlo se může chovat jako částice a zároveň jako vlna). 
Právě zkoumání tohoto jevu se budeme věnovat v tomto seriálu.

Náš příběh začíná na začátku 20. století  u 26letého Alberta Einsteina, který se v té době krom jiných věci pokoušel vysvětlit tzv. \emph{fotoelektrický jev}.

Fotoelektrický jev (\emph{fotoefekt}) spočívá v uvolnění (a následné emitaci) elektronů z obalu atomu po absorpci elektromagnetického záření
 
% doktorská práce (brokolice) - obecná formulace dualismu (vlna -> částice --> částice -> vlna)

% btw 2 || exp. -> důkaz (za jistých podmínek - pozorování)

%       |     )  )  ))         |
%   / - -) ) ) ) ) ))))        |
% -     |  ## # ) ) )          |
%   \ - -) ) ) ) ) ) ) )       |
%       |     )) ) )))         |

% DeBroglieho vlna? ANO (určitě)
% $$ - možná na začátek
% \lambda=\frac{h}{p}
% $$

% 

% Bohrův model -> seriál: Schrödingerův model atomu

% !!! footnote odkaz na výfučtení 5. série 12. ročník !!!
\footnote{Pokud si od nás chcete na toto téma přečíst víc: \url!vyfuk.org/_media/ulohy/r12/s5/vyfucteni5.pdf!}

%\\
%\\
%\textbf{Úloha:}

%<*task>

%\begin{enumerate}[\noindent {}]
%\item otazka
%  \begin{choices}
%    \choice
%  \end{choices}
%\end{enumerate}

%</task>
\end{document}
