\documentclass{../../../../style/mkimain}

\series{4}
\month{květen}
\year{2023}

\begin{document}
%<*header>
\section*{IV.U2 Světla, kamera, Stockholm!}
%</header>
%<*task>

\noindent Za rok 2022 byla udělena Nobelova cena za fyziku Alainu Aspectovi, Johnu Clauserovi a Antonu Zeilingerovi %
konkrétně za experimenty s provázanými fotony. Jejich výsledky potvrdili neplatnost tzv. \emph{Bellových nerovností} u provázaných částic, %
což vede k faktu, že se tyto částice ovlivňují na dálku a to nadsvětelnou rychlostí. %
Je takto ale možné posílat informace rychleji než světelnou rychlostí? %
Pokuste se nastínit zdůvodnění své odpovědi.
%</task> 
\end{document}
