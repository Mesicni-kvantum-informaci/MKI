\documentclass{../../../../style/mkimain}

\series{2}
\month{březen}
\year{2023}

\begin{document}
\section*{II.B Zase ty světla!}
\noindent
Netrpělivý řidič se přibližuje k semaforu, na kterém z dálky vidí svítit červenou.
Nechce zastavovat, a jelikož je fyzikálně vzdělaný, napadne ho zrychlit na takovou rychlost, že místo červené uvidí zelenou.
Vypočítejte rychlost, jakou by se musel pohybovat. $\lambda_\mathrm{R}=\qty{700}{\nm}$ $\lambda_\mathrm{G}=\qty{550}{\nm}$.
\end{document}
