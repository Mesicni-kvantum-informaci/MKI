\documentclass{../../../../style/mkimain}

\series{2}
\month{březen}
\year{2023}

\begin{document}
%<*content>
%<*header>
\section*{II.A Polární záře}
%</header>

% Úvod
Záře viditelná v okolí zemských pólů pojmenovaná podle řecké bohyně úsvitu - \emph{Aurora}.
Snad každý o polární záři již někdy slyšel a někteří šťastlivci dokonce měli možnost
tuto překrásnou podívanou vidět na vlastní oči, avšak málokdo ví jak doopravdy vzniká.
V tomto seriálu se vám pokusíme objasnit fascinující pouť, která stojí za vznikem jednoho
 z nejkouzelnějších přírodních jevů na Zemi.
\\

% Slunce
Náš příběh začíná u nám nejbližší hvězdy, tedy u Slunce. Ve vnější vrstvě sluneční atmosféry
vznikají \emph{koronální smyčky}. Obrovské \uv{trubice} proudící z jedné sluneční skvrny do
druhé jsou tvořené nepředstavitelně horkým a hustým \emph{plazmatem}, které je v tomto tvaru drženo magnetickým polem.
Plazma, tedy ionizovaný plyn natahuje a deformuje toto magnetické pole směrem od Slunce a
dvakrát až třikrát za den se plazmatu podaří oddělit část magnetického pole od~Slunce.
Při \emph{výronu koronální hmoty} (\emph{CME}) se od Slunce oddělí obrovský oblak \footnote[1]{Jak by řekl prof. Kulhánek - chrchel} plazmatu obklopený silným
magnetickým polem složený převážně z protonů, elektronů a alfa částic (jader helia), neboli \emph{plazmoid}.
Ten se pak vydá meziplanetární cestu dlouhou stovky miliónů kilometrů.

% Magnetické pole Země
Po zhruba 18hodinovém letu dorazí \emph{sluneční vítr} k Zemi. Země má díky pohybu tekuté vrstvy
vnějšího jádra složeného převážně z niklu a železa vlastní magnetické pole \emph{dipólového charakteru}.
To neznamená nic jiného, než že se magnetické pole Země chová podobně jako magnetické pole
obyčejného tyčového magnetu, který si všichni nepochybně pamatujeme z hodin fyziky. Magnetické pole Země
se ale od tyčového magnetu liší tím, že je deformované neustálým působením slunečního větru. Sluneční
vítr splošťuje stranu magnetosféry přivrácenou ke Slunci na cca 5 zemských průměrů a naopak
tvaruje odvrácenou stranu do tzv. \emph{magnetického ohonu}, který sahá až do vzdálenosti cca
100 zemských průměrů.

% Polární záře na Zemi
Magnetické pole plazmoidu, jehož siločáry mají opačný směr než zemské magnetické pole, začne
interagovat s magnetickým polem Země. Siločáry magnetického pole Země se spojí se siločarami plazmoidu
a u zemských pólů vytvoří \uv{trychtýře}, zvané \emph{kaspy} \footnote[2]{Ano, opravdu \emph{kaspy},
nikoliv kapsy (jak je občas psáno), vychází z anglického \emph{cusp}}. Jelikož je plazma vázáno na magnetické pole,
budou všechny nabité částice \uv{sledovat} tyto spojené siločáry až k zemským pólům. Elektrony
a protony slunečního větru budou v magnetosféře Země konat hned několik periodických pohybů.
Jednak \emph{gyraci}, tedy oběh okolo magnetických siločar po šroubovici, jednak tzv. \emph{drift},
což je oběh okolo Země, nabité částice tedy postupně střídají siločáry, okolo kterých gyrují a
poslední pohyb, který částice konají je pohyb po siločáře mezi zemskými póly. Elektronu trvá
pouhé 4 sekundy dostat se od jednoho pólu k druhému. Důsledkem tohoto pohybu je jakási
propojenost polárních září Severního a Jižního pólu, tedy \emph{aurory borealis} a \emph{aurory australis}.

% Barva polární záře
Jistě jste si již někdy všimli, že polární záře může mít mnoho různých barev od zelené až
po červenou. Barva polární záře záleží na molekule, které nabité částice slunečního větru
na pólu předají svou energii. V nejvyšších výškách atmosféry je největší koncentrace atomárního
kyslíku \ce{O}, který nejvíce vyzařuje energii na vlnové délce \qty{630}{\nm} (červená). V nižších
vrstvách atmosféry je vysoká koncentrace dusíku \ce{N_2} a dochází k nesčetně mnoho srážkám mezi
molekulami dusíku a kyslíku. Pokud dusík absorbuje energii jako první a pak jí srážkou předá
kyslíku část energie se ztratí a kyslík ji tedy vyzáří na vlnové délce \qty{557.7}{\nm} (zelená).
Většina atomárního kyslíku se nachází v atmosféře \qty{100}{\km} nad povrchem Země a výše. Pod
touto hladinou tedy převládá dusík \ce{N_2}, který sám vyzařuje na vlnové délce \qty{428}{\nm} (modrá).
%</content>
\\
\\
\textbf{Úloha:}
\\
\\
%<*task>
V sériálu jste se dozvěděli o polární záři na Zemi, nyní se zkuste zamyslet, jak je to s polární září na naší
sousední planetě Venuši. Rozhodněte jestli lze v atmosféře Venuše pozorovat jev podobný polární záři na Zemi, pokud ano popište, jak vzniká.
%</task>
\end{document}
