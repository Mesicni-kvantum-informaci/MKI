\documentclass{../../../../style/mkimain}

\series{2}
\month{březen}
\year{2023}

\begin{document}
\ExecuteMetaData[../../problems/problem2-B/problem2-B.tex]{header}
\noindent\ExecuteMetaData[../../problems/problem2-B/problem2-B.tex]{task}
\proborigin{Jindra přemýšlel nad slavným citátem \uv{Two B or not two B?}}.
\klein
Od semaforu se šíří světlo směrem k autu. Jelikož se auto pohybuje, a vlnová délka světla se důsledkem \emph{Dopplerova jevu} zmenšuje.
Pro frekvenci $f_e$ emitovaného světla a frekvenci $f_p$ přijatého světla platí rovnice
$$f_e=f_s\left(1+\frac{v_p}{v}\right),$$
kde $v_p$ je rychlost přijímače, tedy auta, a $v$ je rychlost vlnění, v našem případě světla.
\end{document}

%https://www.christian-doppler.net/cs/doppleruv-jev/
