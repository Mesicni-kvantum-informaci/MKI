\documentclass{../../../../style/mkimain}

\series{2}
\month{březen}
\year{2023}

\begin{document}
\ExecuteMetaData[../../problems/problem2-U2/problem2-U2.tex]{header}
\noindent\ExecuteMetaData[../../problems/problem2-U2/problem2-U2.tex]{task}
\proborigin{}
\klein

V roce 1865 slavný skotský fyzik \emph{James Clerk Maxwell} zformuloval své čtyři rovnice elektromagnetismu,
podle kterých se může elektrické a magnetické pole nést ve formě vlny. %
% ..., podle kterých se může elektrické a magnetické pole šířit ve formě vlny. - co na to říkáte? -- Vojta
Zjistilo se rovněž, že tato vlna nápadně odpovídá světlu.
To, co je ale na Maxwellových rovnvicích zajímavé a podivné, je tvrzení,
že světlo jakožto nosič elektromagnetického pole se šíří konstantní rychlostí nezávisle na tom, z jaké soustavy se na něj díváme.
Na základech této Maxwellovy teorie byly vystavěny tzv. \emph{Lorentzovy transformace}
a následně i věhlasná Einsteinova speciální teorie relativity.

% Lorentzovy transformace v množném čísle?, also stejně jako předtím není explicitně zmíněna správná odpověď -- Vojta
\end{document}
