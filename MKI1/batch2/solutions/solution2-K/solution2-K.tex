\documentclass{../../../../style/mkimain}

\series{2}
\month{březen}
\year{2023}

\begin{document}
\ExecuteMetaData[../../problems/problem2-K/problem2-K.tex]{header}
\noindent\ExecuteMetaData[../../problems/problem2-K/problem2-K.tex]{task}
\proborigin{Michal hledá všechny možné způsoby, jak by mohl zazářit.}
\klein

Vlnová délka, na které těleso (z původní teorie absolutně černé) vyzařuje nejvíce, je \textbf{nepřímo úměrná} jeho teplotě.
Tento fakt se dá dokázat matematicky hledáním maxima funkce spektrální intenzity (pomocí matematické operace zvané \emph{derivování}).
Protože se jedná o docela signifikantní poznatek, vysloužil si vlastní název, \emph{Wienův posunovací zákon}.

V přírodě ho lze pozorovat zejména ve vesmíru, například barva hvězd je bezprostředně určena jejich teplotou.
Modré hvězdy jsou teplejší než oranžové a to právě proto, že modré barvě odpovídá kratší vlnová délka než oranžové.
V astronomii proto lze pozorováním barvy hvězd relativně snadno určit jejich povrchovou teplotu.

% Kupříkladu lze říct, že modré hvězdy jsou teplejší než oranžové a to právě proto, že modré barvě odpovídá kratší vlnová délka než oranžové.
% V astronomii se proto pozorováním barvy hvězd snadno určuje jejich povrchová teplota.
% trochu jsem to přeformuloval -- Vojta

Zákon vyzařování se projevuje i v jedné v současné době probírané problematice, kterou je skleníkový efekt.
Nepochybně jste už někdy slyšeli, jak tento jev funguje. V jednoduchém podání sluneční světlo projde atmosférou Země,
ale jeho energie se už poté nevrátí zpátky do vesmírného prostoru. Jak je to možné? Energii ze Slunce pohltí zemský povrch, který se tak ohřívá.
Podle Planckova zákona musí i samotná Země vyzařovat. Ovšem na mnohem delší vlnové délce, než je viditelné světlo ze Slunce,
protože Země má porvchovou teplotu mnohokrát nižší než Slunce. Naše planeta ve výsledku vrací přijatou enrgii v infračervené oblasti,
jenže právě na takových vlnových délkách absorbují záření ony skleníkové plyny
(původ toho děje tkví v kvantové chemii a tzv. \emph{Ramanově spektroskopii}) a udržují tím energii v atmosféře,
čímž se planeta rychle otepluje\dots
% mám takový pocit, že po \dots se nepíše tečka, also nechceme přidat odkaz na nějaký zajímavý zdroj na Ramana pro lidi co by to zajímalo?

V poslední řadě by bylo hezké uvést něco, co všichni nepochybně znáte. Infračervená kamera, přístroj umožňující vidění ve tmě,
též zvaný jako termovizní kamera. Funguje přesně na principu sledování infračeveného záření, tedy záření, které člověk emituje.
Jeho vlnovou délku přepočítá na teplotu tělesa a vykreslí nám hezký obraz rozložení teploty v okolním prostoru.
% Funguje přesně na principu sledování infračeveného záření, záření, které člověk emituje. - docela divně formulovaná věta
% ale jinak pěkný vzorák -- Vojta
\end{document}
