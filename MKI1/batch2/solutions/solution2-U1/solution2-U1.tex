\documentclass{../../../../style/mkimain}

\series{2}
\month{březen}
\year{2023}

\begin{document}
\ExecuteMetaData[../../problems/problem2-U1/problem2-U1.tex]{header}
\noindent\ExecuteMetaData[../../problems/problem2-U1/problem2-U1.tex]{task}
\proborigin{}
\klein
V čím nižší úhlové výšce (výšce nad obzorem) světlo vzhledem k pozorovateli přichází, tím delší dráhu v atmosféře musí urazit.
Jelikož tedy urazí delší dráhu, světlo se více rozptýlí. Je tedy jasné, že správná odpověd je a).
\end{document}
