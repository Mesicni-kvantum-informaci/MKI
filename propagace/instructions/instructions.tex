\documentclass[12pt]{article}
\usepackage{import, fancyhdr, layout, graphicx}
\usepackage{blindtext, mdframed, float}
\usepackage{siunitx}
\usepackage{standalone}
\sisetup{output-decimal-marker = {,}}
\usepackage[czech]{babel}

\hoffset=-0,71cm
\voffset=1cm
\topmargin=-2cm
%\headheight=0cm
\headsep=1cm
\textheight=22cm
\textwidth=18cm
\footskip=-1cm
\oddsidemargin=0pt

    %záhalví 
\pagestyle{fancy}
\fancyhead[R]{2023}
\fancyhead[C]{Měsíční kvantum informací}
\fancyhead[L]{Pravidla}

    %záhlaví první strany
\fancypagestyle{firststyle}
{
   \fancyhf{}
   \renewcommand{\headrulewidth}{0pt}
}
\usepackage{xcolor, hyperref}

\begin{document}
\section*{Pravidla}
\begin{itemize}
    \item K řešení můžete používat libovolnou literaturu, internet, kamarády\dots nic není zakázané
    \item Svá řešení sepisujte na libovolný list papíru s čísly úloh a odpovědmi k nim, aby bylo rozpoznatelné, která odpověď patří k jaké otázce
    \item V případě přiřazovaček a poznávaček využijte námi připravené číslování obrázků
    \item Kromě odpovědí můžete k úlohám psát i různé poznámky, např. pokud se vám zdá, že máme někde chybu, nebo vás napadla zajímavá myšlenka během řešení, neváhejte se podělit! Může vám to vynést i nějaké bonusové body\dots
    \item List s řešením v rohu podepište a uveďte třídu 
    \item Svá řešení odevzdávejte na vrátnici nebo zasílejte elektronicky na náš mail: 
    
    \color{blue}
    \href{mailto:kvantuminformaci@gmail.com}{kvantuminformaci@gmail.com}
    \color{black}
    \item Řešení můžete odevzdávat nejpozději do \textbf{posledního pracovního dne} v měsíci.
    \item Nemusíte řešit všechno! Stačí si vybrat pár úloh, které se vám líbí (a ani jednotlivou úlohu nemusíte vyřešit celou)
    \item Za každou vyřešenou sérii si můžete vysloužit jedničku z fyziky.
    \item Soutěž se bude hodnotit ve čtyřech kategoriích:
    \begin{enumerate}
        \item \textbf{kategorie D}: prima a sekunda osmiletého gymnázia
        \item \textbf{kategorie C}: tercie a kvarta osmiletého gymnázia a odpovídající ročníky šestiletého gymnázia
        \item \textbf{kategorie B}: kvinta a sexta osmiletého gymnázia a odpovídající ročníky šestiletého gymnázia
        \item \textbf{kategorie A}: septima a oktáva osmiletého gymnázia a odpovídající ročníky šestiletého gymnázia
    \end{enumerate}
    \item Na konci roku proběhne slavnostní vyhlášení, nejlepší řešitelé každé kategorie obdrží odměnu
\end{itemize} 
\end{document}