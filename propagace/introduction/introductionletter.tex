\documentclass{article}

\begin{document}
\section*{MKi? MKi!}
Vážení řešitelé,
\\\\
Do rukou se Vám dostává první série našeho nového korespondenčního semináře MKi – \textit{Měsíční kvantum informací}!
Je pro nás čest mít tu možnost Vám nyní představit tuto soutěž,
při které máte možnost zaměstnat svou bystrou mysl něčím jiným než jen klasickými příklady z hodin fyziky.
Už Vás unavuje to neustálé ťukání do kalkulaček? Při řešení MKi můžete všechny počítací nástroje rovnou zahodit!
Ačkoliv se jedná o fyzikální úlohy, není potřeba cokoliv počítat.
MKi se zaměřuje právě na Vaši schopnost uvažovat nikoliv na schopnost rozumět rovnicím.
Náš seminář tedy klade velkou váhu na samou podstatu fyziky, což je chtít uvažovat. Tak jestli Vás fyzika baví i nebaví, neváhejte MKi zkusit.
Zaručujeme zcela novou perspektivu učení se přírodním vědám.
\\\\
Těšíme se na Vaše řešení!
\\
Tým MKi

\section*{Podrobné informace}

\begin{enumerate}
    \item Soutěž probíhá celý rok a sestává z jednotlivých sérií
    \item Na řešení každé série máte jeden měsíc
    \item Každá série sestává ze:
    \begin{itemize}
        \item tří obecných úloh
        \item dvou seriálů
        \item úloh k daným seriálům
        \item jedné bonusové úlohy
    \end{itemize}
    \item \textbf{Obecné úlohy} jsou inspirované lehčí tematikou zpravidla na úrovni nižšího gymnázia a nevyžadují žádné počítání (nejčastěji mají formu poznávačky, přiřazovčky, kroužkovacího testu\dots)
    \item \textbf{Seriály} jsou krátké odborné texty z populární fyziky na dvě témata: \textit{Astronomie} a \textit{Kvantový svět}
    \item \textbf{Úlohy k seriálům} jsou úlohy podobného typu jako obecné úlohy, akorát se tematicky vztahují k daným seriálům, a je proto rozumné si před jejich řešením seriály přečíst
    \item \textbf{Bonusová úloha} je taková menší výzva pro lidi, které by mrzelo, kdyby si nezapočítali\dots
    \item Ke každé úloze je připsán maximální počet bodů, který za ni můžete dostat
    \item Pokud se odhodláte MKi řešit, jednička z fyziky vás nemine
    \item Na konci roku proběhne slavnostní vyhlášení pro všechny kategorie
    \end{enumerate}

\end{document}